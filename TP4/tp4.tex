\documentclass{exam}

\usepackage[utf8]{inputenc}
\usepackage{amsmath}
\usepackage{amssymb}
\usepackage{amsthm}
\usepackage{amssymb}
\usepackage{amsfonts}
\usepackage[francais]{babel}
\usepackage{fancyvrb}
\usepackage{hyperref}

\title{TP4\\ Arithmétique}

\begin{document}
\maketitle

\begin{questions}
\question
{\bf PGCD de deux entiers, identité de Bézout, algorithme d'Euclide.}
\begin{parts}
\part
Rappeler la définiton du pgcd de deux entiers relatifs ainsi que l'énoncé de l'identité de Bézout.
\part
Chercher comment on obtient en python le quotient et le reste de la division euclidienne d'un entier par un entier.
A l'aide de l'algorithme d'Euclide étendu \url{https://en.wikipedia.org/wiki/Extended_Euclidean_algorithm}, calculer le pgcd $d$ des deux nombres $a = 4864, b = 3458$ ainsi que les coefficients de Bézout $x, y$ tels que $xa + yb = d$.
\part
Ecrire une fonction python \texttt{euclide} qui prend en arguments deux entiers $a, b$ et qui renvoie $x, y, d$, où $d$ est le pgcd de $a, b$ et $x, y$ les coefficients de Bézout.

\end{parts}


\question

\question

\end{questions}

\end{document}

%%% Local Variables:
%%% mode: latex
%%% TeX-master: t
%%% End:
