\documentclass{exam}

\usepackage[utf8]{inputenc}
\usepackage{amsmath}
\usepackage{amssymb}
\usepackage{amsthm}
\usepackage{amssymb}
\usepackage{amsfonts}
\usepackage[francais]{babel}
\usepackage{fancyvrb}
\usepackage{hyperref}

\title{TP3\\ Equations différentielles}

\begin{document}
\maketitle


Dans ce TP, on présente quelques méthodes pour intégrer numériquement sur un intervalle $t \in [0, T]$, une équation différentielle ordinaire - dite EDO - c'est-à-dire de la forme
$$u'(t) =  f(t, u(t)),\quad u(0) = u_0$$
La fonction $f$, qui prend en argument un couple de réels et qui renvoie un réel, ainsi que la valeur initiale $u_0$ sont donnés; la fonction $u$, qui prend un réel en argument et qui renvoie un réel, est inconnue; c'est la solution cherchée.

\begin{questions}

\question
Exemple : considérons l'EDO très simple
$$u'(t) =  -u(t), \quad u(0)  =  1.0$$
Le problème est ici de déterminer la solution $u$.
\begin{parts}
\part
Que vaut la fonction $f$ dans cet exemple
\part
Calculer à la main la solution $u$ de l'EDO ci-dessus, et la dessiner sur papier; on prendra comme intervalle $t \in [0, 2]$.
\part
Sur ordinateur, représenter graphiquement $u$ à l'aide du module \texttt{matplotlib}; on créera le vecteur d'abscisses au moyen de \texttt{numpy.linspace} et on sauvera la figure au format \texttt{png} dans le répertoire courant.
\end{parts}

\question
Calcul approché de $u$ au moyen de la {\bf méthode d'Euler}.

On divise l'intervalle $[0, T]$ en $n$ parties égales, puis on pose $h=T/n$, ce qui fournit une subdivision $t_0 = 0, t_1 = h, t_2 = 2h, \cdots, t_n = T$. A partir de là veut calculer, approximativement, la valeur de $u$ aux points $t_k$ de la subdivision. Voici l'idée :\\
Supposons que l'on connaisse, pour un indice $k$, une approximation $u_k$ de la valeur exacte $u(t_k)$; comment alors calculer une approximation $u_{k+1}$ de la valeur exacte $u(t_{k+1})$? Réponse :\\
On écrit $u'(t_k) \approx \frac{u(t_{k+1})-u(t_k)}{h}$, approximation d'autant meilleure que $h$ est petit;\\
 on a donc 
 $\frac{u(t_{k+1})-u(t_k)}{h} \approx u'(t_k) = f(t_k, u(t_k))$, puis $u(t_{k+1}) \approx u(t_k) + hf(t_k, u(t_k))$\\
  et enfin 
  $u(t_{k+1}) \approx u_k + hf(t_k, u_k)$. On a donc trouvé une approximation $u_{k+1}$ de la valeur exacte $u(t_{k+1})$, à savoir 
$$u_{k+1} = u_k + hf(t_k, u_k)$$
Cette formule permet, par récurrence, de calculer une suite d'approximations $u_k$, à condition que l'on connaisse une première approximation $u_0$ de $u(t_0)$; mais pour $u_0$ on peut bien sûr prendre la valeur de la condition initiale donnée par l'EDO.
\begin{parts}
\part
Calculer la suite $u_k$, définie par la méthode d'Euler, pour l'exemple ci-dessus; on prendra $T=2.0$ et $n=10$.
\part
Représenter sur le même graphique la solution exacte $u$ et les points $(t_k, u_k)$; faire le travail sur papier et numériquement à l'aide de \texttt{matplotlib}.
\end{parts}

\question
Ecrire une fonction python \texttt{euler} qui prend en arguments une foncion $f$ de deux variables scalaires $t, u$, une valeur initiale $u_0$, un réel $T$, un entier $n$, et qui renvoie la liste $tt$ des $t_k$ et la liste $uu$ des $u_k$, valeurs obtenues par la méthode d'Euler (alternativement, on pourra renvoyer $tt$ et $uu$ sous forme de \texttt{numpy arrays}.
\question
\begin{parts}
\part
Appliquer la fonction \texttt{euler} à l'exemple précédentci-dessus (on aura besoin d'écrire la fonction $f$ de l'EDO - fonction de deux variables - dans une fonction python) et retrouver les résultats précédents.
\part
Reprendre le travail précédent avec les EDOs suivantes (attention, certaines équations peuvent être facilement intégrées à la main, d'autres non).
\begin{align} 
u'(t) &=  -u(t) + t, & u(0)  &=  1.0 \\ 
u'(t) &=  u^2(t), & u(0)  &=  1.0 \\ 
u'(t) &=  u^2(t) - t, & u(0)  &=  1.0
\end{align}
\end{parts}

\question
Nous montrons maintenant comment intégrer une EDO d'ordre supérieur. 
Prenons l'exemple de l'oscillateur harmonique (modélisation du ressort) qui est une EDO d'ordre $2$
$$u''(t) + \omega^2 u(t) = 0,\quad u(0) = u_0, u'(0) = v_0$$
où $\omega, u_0, v_0$ sont des scalaires donnés; $\omega$ s'appelle la pulsation, $u_0$ la position initiale, $v_0$ la vitesse initiale.



\end{questions}
\end{document}

%%% Local Variables:
%%% mode: latex
%%% TeX-master: t
%%% End:
