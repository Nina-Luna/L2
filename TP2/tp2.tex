\documentclass{exam}

\usepackage[utf8]{inputenc}
\usepackage{amsthm}
\usepackage{amssymb}
\usepackage{amsfonts}
\usepackage[francais]{babel}
\usepackage{fancyvrb}
\usepackage{hyperref}

\title{TP2\\ Intégration numérique}

\begin{document}
\maketitle


Dans ce TP, on va apprendre quelques méthodes numériques pour calculer, approximativement, l'intégrale d'une fonction sur un intervalle compact (borné, fermé).

\begin{questions}

\question
Création d'une fonction simple et représentation graphique.
\begin{parts}
\part
Ecrire la fonction $f(x) = 1 - x^2$ sous forme d'une fonction python. 
\part
Représentation graphique sur l'intervalle $\left[0, 1\right]$.
\part
Calculer, à la main, l'intégrale $I$ de cette fonction sur l'intervalle donné.
\end{parts}

\question
Calcul approché de $I$ au moyen de la {\bf méthode du point milieu}.
\begin{parts}
\part
Représenter graphiquement $f$ sur papier, repère orthonormé, unité = $8$ cm.
\part
On définit une subdivision régulière $x_0, \cdots, x_4$ de l'intervalle d'intégration $[0, 1]$ en $n=4$ parts égales (on a donc $x_0 = 0, x_1 = 0.25, \cdots, x_4 = 1$. On note $c_1, \cdots, c_4$ les milieux de ces sous-intervalles. Pour chaque $k = 1, \cdots, 4$, dessiner le rectangle de base $[x_{k-1}, x_k]$ et de hauteur $f(c_k)$, et calculer sa surface $s_k$.
\part
La méthode du point milieu consiste à prendre la somme $S_4 = \sum\limits_{i=1}^4 s_k$ comme approximation de $I$. Calculer l'erreur commise $\vert S_4 - I \vert$.
\part
Importer le module \texttt{time}. Utiliser la fonction \texttt{clock()} du module \texttt{time} pour mesurer le temps de calcul de votre intégrale.
\part
Recommencer les calculs précédents pour $n = 10^k$, k variant de $1$ à $6$, et remplir, manuellement, le tableau ci-dessous:
\begin{center}
\begin{tabular}{r | c | c}
{n} & erreur & temps (sec.)\\
\hline
$10$ & {} & {}\\
$100$ & {} & {}\\
$1000$ & {} & {}\\
$10000$ & {} & {}\\
$100000$ & {} & {}\\
$1000000$ & {} & {}
\end{tabular}
\end{center}
\part
En python, recréer automatiquement le tableau obtenu au moyen d'une boucle \texttt{for}. Utiliser  les fonction \texttt{print} pour voir le tableau à l'écran et \texttt{write} pour écrire le tableau dans un fichier texte.\\
Voir documentation à l'adresse \url{https://docs.python.org/3/tutorial/inputoutput.html}.\\
Exemple de tableau produit automatiquement et envoyé à l'écran:
\VerbatimInput{table.txt}

\end{parts}


\end{questions}

\end{document}



