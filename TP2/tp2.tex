\documentclass{exam}

\usepackage[utf8]{inputenc}
\usepackage{amsthm}
\usepackage{amssymb}
\usepackage{amsfonts}
\usepackage[francais]{babel}
\usepackage{fancyvrb}
\usepackage{hyperref}

\title{TP2\\ Intégration numérique}

\begin{document}
\maketitle


Dans ce TP, on présente quelques méthodes numériques pour calculer, approximativement, l'intégrale d'une fonction sur un intervalle compact (borné, fermé).

\begin{questions}

\question
Création d'une fonction simple et représentation graphique [Python : module \texttt{math}; Maths : calcul exact d'une intégrale].
\begin{parts}
\part
Ecrire la fonction $f(x) = \sqrt{1 - x^2}$ 
sous forme d'une fonction python (on utilisera la fonction \texttt{sqrt()} du module \texttt{math}).
\part
Représentation graphique sur l'intervalle $[-0.5, 0.5]$.
\part
Calculer, à la main, l'intégrale $I$ de cette fonction sur l'intervalle donné.
\end{parts}

\question
Calcul approché de $I$ au moyen de la {\bf méthode du point milieu} [Python : module \texttt{time}, création et écriture d'un tableau de résultats dans un fichier texte au moyen des fonctions \texttt{open()} et \texttt{write()}; Maths : approximation d'une fonction par une fonction en escalier, approximation numérique d'une intégrale].
\begin{parts}
\part
Représenter graphiquement $f$ sur papier, repère orthonormé, unité = $8$ cm.
\part
On définit une subdivision régulière $x_0, \cdots, x_4$ de l'intervalle d'intégration $[0, 1]$ en $n=4$ parts égales (on a donc $x_0 = -0.5, x_1 = -0.25, \cdots, x_4 = 0.5$. On note $c_1, \cdots, c_4$ les milieux de ces sous-intervalles. Pour chaque $k = 1, \cdots, 4$, dessiner le {\bf rectangle} de base $[x_{k-1}, x_k]$ et de hauteur $f(c_k)$, et calculer sa surface $s_k$. La méthode du point milieu consiste à prendre la somme $S_4 = \sum\limits_{i=1}^4 s_k$ comme approximation de $I$\part
Calculer l'erreur commise $\vert S_4 - I \vert$.
\part
Utiliser la fonction \texttt{clock()} du module \texttt{time} pour mesurer le temps de calcul de votre intégrale.
\part
Ecrire une fonction python \texttt{point\_milieu} qui prend en arguments une foncion $f$, des bornes $a, b$ et un entier $n$ et qui renvoie l'intégrale approchée de $f$ sur $[a, b]$ au moyen de la méthode du point milieu.
\part
Tester \texttt{point\_milieu} avec $f(x) = \sqrt{1 - x^2}$, $a=-0.5, b=0.5$ et $n = 10^k$, k variant de $1$ à $6$, puis remplir manuellement le tableau ci-dessous:
\begin{center}
\begin{tabular}{r | c | c}
{n} & erreur & temps (sec.)\\
\hline
$10$ & {} & {}\\
$100$ & {} & {}\\
$1000$ & {} & {}\\
$10000$ & {} & {}\\
$100000$ & {} & {}\\
$1000000$ & {} & {}
\end{tabular}
\end{center}
\part
En python, recréer automatiquement le tableau obtenu au moyen d'une boucle \texttt{for}. Utiliser  les fonction \texttt{print} pour voir le tableau à l'écran et \texttt{write} pour écrire le tableau dans un fichier texte.\\
Voir documentation à l'adresse \url{https://docs.python.org/3/tutorial/inputoutput.html}.\\
Exemple de tableau produit automatiquement en python:
\VerbatimInput{table.txt}
\end{parts}

\question
Dans la méthode du point milieu, on a approximé la fonction $f$ sur l'intervalle $[x_{k-1}, x_k]$ par la fonction constante prenant la même valeur que $f$ en $c_k$, le milieu de $[x_{k-1}, x_k]$; dans la {\bf méthode du trapèze}, on approxime $f$ sur cet intervalle par la fonction affine qui prend les mêmes valeurs que $f$ en $x_{k-1}$ et $x_k$.\\
Reprendre le travail précédent, mais avec la méthode du trapèze.

\question
Dans la méthode du point milieu, on a approximé $f$ sur l'intervalle $[x_{k-1}, x_k]$ par une fonction constante - polynôme de degré $0$; dans la méthode du trapèze, on a approximé $f$ sur cet intervalle par une fonction affine - polynôme de degré $1$; dans la {\bf méthode de Simpson}, on approxime $f$ sur l'intervalle $[x_{k-1}, x_k]$ par le polynôme de degré $2$ qui prend les mêmes valeurs que $f$ en $x_{k-1}$, $c_k$ et $x_k$.\\
Reprendre le travail précédent, mais avec la méthode de Simpson.

\question
Comparer les trois méthodes précédentes.

\question
La {\bf méthode de Monte-Carlo} est intéressante pour calculer des surfaces, des volumes, etc., et on pourrait donc l'appliquer au problème ci-dessus. Présentons la méthode en modifiant un peu la situation précédente [Python : module \texttt{numpy.random}; Maths : distribution de probabilité].
\begin{parts}
\part
A l'aide de \texttt{matplotlib.pyplot}, dessiner le cercle unité (centré à l'origine et de rayon $1$). Quelle est la surface du disque unité ?
\part
En ligne de commande, à l'aide de la fonction \texttt{numpy.random.rand()}, générer quelques valeurs aléatoires suivant la distribution de loi uniforme sur $[0, 1]$. Même exercice avec une distribution de loi uniforme sur $[-1, 1]$.
\part
Générer $N=1000$ points suivant la distribution uniforme sur le carré $[-1, 1]^2$. Faire apparaitre ces $N$ points sur le graphique précédent, en rouge les points intérieurs au disque unité, et en vert les points extérieurs au disque.
\part
Compter le nombre $I$ de points intérieurs et le nombre $E$ de points extérieurs. La méthode de Monte-Carlo consiste à prendre le rapport $\frac{I}{N}$ comme approximation de la surface $S$ du disque.
\part
Calculer l'erreur commise $\vert \frac{I}{N} - S \vert$.
\part
Utiliser la fonction \texttt{clock()} du module \texttt{time} pour mesurer le temps de calcul de votre intégrale.
\part
Ecrire une fonction python \texttt{Monte\_Carlo} qui prend en arguments un entier $N$ et qui renvoie une approximation de la surface du disque unité au moyen de la méthode de Monte-Carlo.
\part
Tester \texttt{Monte\_Carlo} avec $N = 10^k$, k variant de $1$ à $6$, puis créer automatiquement, et l'enregistrer dans un fichier texte, un tableau de résutats du modèle ci-dessous:
\begin{center}
\begin{tabular}{r | c | c}
{N} & erreur & temps (sec.)\\
\hline
$10$ & {} & {}\\
$100$ & {} & {}\\
$1000$ & {} & {}\\
$10000$ & {} & {}\\
$100000$ & {} & {}\\
$1000000$ & {} & {}
\end{tabular}
\end{center}


\end{parts}



\end{questions}

\end{document}



